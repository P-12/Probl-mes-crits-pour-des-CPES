\documentclass[10pt,a4paper]{article}
\usepackage[utf8]{inputenc}  
\usepackage[T1]{fontenc}       
\usepackage[francais]{babel}
\usepackage{fullpage}
%\usepackage{euler}
\usepackage{amsmath}
%\usepackage{framed}
\usepackage{amsfonts}
\usepackage{amssymb}
\usepackage{pifont}
\usepackage{mathrsfs}
\usepackage{graphicx}
\usepackage[a4paper]{geometry}
\geometry{hscale=0.86,vscale=0.87,centering}

\pagestyle{empty}

%\newcommand{\Titre}[4]{\noindent \textsc{#1}} \hfill \textbf{\textsc{#2}}\\ #3 \hfill \emph{#4}\\  \hrule\vspace{\baselineskip}}

\newcommand{\Titre}[3]{\begin{center} {\LARGE\textbf{\textsc{#1}}}\\ #2 \hfill \emph{#3} \\  \hrule\vspace{\baselineskip}\end{center}}

\begin{document}


\Titre{DM de Maths}{Asp J.Buet}{3 décembre 2013}
\thispagestyle{plain}
\pagestyle{plain}


%\title{DM de Mathématiques}
%\date{19 novembre 2013}
%\author{Asp. J.Buet}
%\maketitle{}


\section{Etude de suites et de fonctions}
Soit $x \in \mathbb{R}$.
On considère la suite $(u_n(x))_{n\in \mathbb{N}}$ définie par $u_n(x)=\sum\limits_{k=0}^nx^k$. Il s'agit d'une suite qui dépend d'un paramètre $x$
 mais dont les indices varient bien sur $\mathbb{N}$. On peut également voir ceci comme une suite dont chaque élément n'est plus un nombre, mais
 une fonction de $x$. On ne prendra pas en compte dans la suite de ce sujet le côté ``fonction'' de chaque terme de la suite (pas de notion de
 continuité ou de dérivée) mais on verra plutôt ceci comme une famille de suites composée des suites $(u_n(0))$, $(u_n(1/2))$, $(u_n(\pi))$, etc.
\begin{enumerate}
\item Calculez $u_1(1/2), u_2(1/2), u_1(-1), u_2(-1)$.
\item On introduit la suite $(v_k(x))_{k \in \mathbb{N}}$ définie par $v_k(x)=x^k$
\begin{enumerate}
\item Quelle est la nature de la suite $(v_k(x))_{k \in \mathbb{N}}$\,? (Arithmétique, géométrique, arithmético-géométrique...)
\item Discutez selon les valeurs de $x$ de la limite de la suite $(v_k(x))_{k \in \mathbb{N}}$.
\item Dédusiez-en des valeurs pour lesquelles la suite $(u_n(x))_{n \in \mathbb{N}}$ ne peut pas avoir de limite finie.
\end{enumerate}
\item Explicitez $u_n$ en fonction de $n$ et de $x$ (sans symbole $\sum$).
\item Pour quelles valeurs de $x$ la suite $(u_n(x))_{n\in\mathbb{N}}$ converge-t-elle\,?
\item Soit $x \in ]-1,1[$, donnez la limite de $(u_n(x))_{n\in\mathbb{N}}$ en fonction de $x$.
\item On pose $f(x)=\frac{1}{1-x}$. Quel est le plus grand domaine sur lequel on peut définir $f$\,? On le considérera désormais comme son
domaine de définition.
\item On introduit la fonction $g$ définie par :
$$\begin{array}{ccccc}
g ~:& ]-1,1[ & \longrightarrow & \mathbb{R} \\
& x & \longmapsto & f(x) \\
\end{array}$$
On l'appelle la restriction de $f$ à $]-1,1[$. Donnez une autre expression de $g$ à l'aide d'une somme infinie (la notation $\sum\limits_{k=0}^{+\infty}$ est 
équivalente à $\lim\limits_{n\to+\infty}\sum\limits_{k=0}^n$). Peut-on dire que cette nouvelle
expression convient pour $f$\,? Pourquoi\,?
\end{enumerate}

\section{Autour de la convergence}
Soit $(a_n)_{n\in\mathbb{N}}$ une suite de complexes telle que $\lim\limits_{n\to+\infty}a_n=1$
Soit $x\in\mathbb{R}$, on considère la suite de fonctions $(f_n(x))_{n\in\mathbb{N}}$ définie par $f_n(x)=\sum\limits_{k=0}^n a_k x^k$. Il s'agit du même objet
que celui introduit à la partie précédente, c'est une suite dont les indices varient sur $\mathbb{N}$ et qui dépend d'un paramètre $x$.
\begin{enumerate}
\item Montrez que la suite $(t_n(x))_{n\in\mathbb{N}}$ définie par $t_n(x)=\sum\limits_{k=0}^n \mid a_k x^k\mid$ est croissante.
\item Montrez qu'il existe $x_0 \in \mathbb{R}$ tel que $(t_n(x_0))$ converge (on pourra montrer par exemple que $(t_n(1/2))$ converge).
\item Pour une suite $(\alpha_n)_{n\in\mathbb{N}}$ réelle, on pose $\alpha_n^+=max(\alpha_n,0)$ et $\alpha_n^-=max(-\alpha_n,0)$. Montrez que 
$\alpha_n=\alpha_n^+ - \alpha_n^-$ et que $\mid \alpha_n \mid = \alpha_n^++\alpha_n^-$.
\item On considère une suite $(\alpha_k)_{k\in\mathbb{N}}$ réelle quelconque. Montrez que si 
$(\sum\limits_{k=0}^n\mid \alpha_k\mid)_{n\in\mathbb{N}}$ converge alors  
$(\sum\limits_{k=0}^n \alpha_k^+)_{n\in\mathbb{N}}$ et $(\sum\limits_{k=0}^n\alpha_k^-)_{n\in\mathbb{N}}$ convergent. 
\item En déduite que si la suite  $(\sum\limits_{k=0}^n\mid \alpha_k\mid)_{n\in\mathbb{N}}$ converge alors la suite 
$(\sum\limits_{k=0}^n \alpha_k)_{n\in\mathbb{N}}$ converge.
\item On rappelle qu'une suite complexe $(b_n)_{n\in\mathbb{N}}$ peut s'écrire de la forme $b_n=\beta_n + i\gamma_n$ où 
$\beta_n$ et $\gamma_n$ sont deux réels.
Une telle suite $(b_n)_{n\in\mathbb{N}}$ converge si et seulement si les deux suites $(\beta_n)_{n\in\mathbb{N}}$ et 
$(\gamma_n)_{n\in\mathbb{N}}$ convergent. En utilisant le résultat précédent sur les suites réelles et le fait que $|Re(b_n)|\leqslant|b_n|$
et $|Im(b_n)|\leqslant|b_n|$, en déduire que si 
$(\sum\limits_{k=0}^n\mid b_k\mid)_{n\in\mathbb{N}}$ converge alors la suite $(\sum\limits_{k=0}^n b_k)_{n\in\mathbb{N}}$ converge.
\item En déduire que $(f_n(x_0))_{n\in\mathbb{N}}$  converge.
\end{enumerate}

\section{Disque de convergence}
On reprend les notations de la partie précédente.
\begin{enumerate}
\item En écrivant que $\mid a_k x^k\mid=\mid a_k x_0^k\mid.\mid\frac{x}{x_0}\mid^k$, montrez que $\forall x \in \mathbb{R}, \mid x\mid < \mid x_0\mid 
\Rightarrow (f_n(x_0))_{n\in\mathbb{N}} \text{ converge}$.
\item De même pour $z \in \mathbb{C}$ avec $\mid z\mid < \mid x_0\mid.$

On note $f(x)$ la limite de la suite $(f_n(x))_{n\in\mathbb{N}}$ quand elle existe. Le domaine de définition de $f$ correspond donc à l'ensemble des $x\in
\mathbb{R}$ pour lesquels $(f_n(x))_{n\in\mathbb{N}}$ converge.

On vient donc de montrer que l'intervalle sur lequel $f$ est définie est centré sur $0$ : si $f$ est définie en $x_0$, elle est définie pour tous les réels 
$x$ tels que $\mid x\mid<\mid x_0 \mid$.
\item Eclairez le problème rencontré à la question 7 de la première partie entre le domaine de définition de $f$ et celui de $g$.
\end{enumerate}

\end{document}
