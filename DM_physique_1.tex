\documentclass[10pt,a4paper]{article}
\usepackage[utf8]{inputenc}  
\usepackage[T1]{fontenc}       
\usepackage[francais]{babel}
\usepackage{fullpage}
%\usepackage{euler}
\usepackage{amsmath}
%\usepackage{framed}
\usepackage{amsfonts}
\usepackage{amssymb}
\usepackage{pifont}
\usepackage{mathrsfs}
\usepackage{graphicx}
\usepackage{wasysym}
\usepackage{pstricks-add}
\usepackage[squaren,Gray]{SIunits}
\usepackage[a4paper]{geometry}
\geometry{hscale=0.86,vscale=0.87,centering}



\pagestyle{empty}

%\newcommand{\Titre}[4]{\noindent \textsc{#1}} \hfill \textbf{\textsc{#2}}\\ #3 \hfill \emph{#4}\\  \hrule\vspace{\baselineskip}}

\newcommand{\Titre}[3]{\begin{center} {\LARGE\textbf{\textsc{#1}}}\\ #2 \hfill \emph{#3} \\  \hrule\vspace{\baselineskip}\end{center}}

\begin{document}

\Titre{DM de Physique}{Asp J.Buet}{3 décembre 2013}
\thispagestyle{plain}
\pagestyle{plain}


\part*{L'appareil photographique}

\bigskip
\section{Optique d'un appareil photo}

L'objectif d'un appareil photo est modélisé par une lentille mince convergente de distance focale $f'_{obj}$, accolée à un diaphragrame circulaire
de diamètre $D$. Les axes de la lentille et du diaphragme sont confondus.

\begin{figure}[h!]
\center
\includegraphics[scale=0.55]{appareil_photo.png}
\end{figure}
La lentille est utilisée dans les conditions de Gauss.

On modélise la pellicule par une matrice de photorécepteurs carrés de côté $\varepsilon$, chacun de ces carrés correspondant à un pixel.

Pour qu'un objet soit perçu net, il faut que l'image d'un point soit comprise à l'intérieur du récepteur carré, même si cette image n'est pas ponctuelle.
\begin{enumerate}
\item Enoncer les conditions de Gauss.
\item La mise au point étant faite à l'infini, quelle est la distance entre l'objectif et le plan de la pellicule\,?
\smallskip
\item \textbf{Pouvoir de résolution}

Pour deux objets à l'infini (on règle l'appareil photo convenablement), quel est l'angle minimum entre ces deux objets pour qu'ils soient perçus 
distinctement sur la photo\,?
\smallskip
\item \textbf{Profondeur de champ}

En fonction de la résolution (de la taille des pixels) de la pellicule, il y a une plage de distances objet-objectif pour lesquelles l'image apparaît nette
sur la pellicule. La longueur de cette plage est appelée profondeur de champ.

Pour un $\varepsilon$ donné, une focale $f'_{obj}$ et une distance $L$ entre l'objectif et la pellicule, donner la distance à laquelle se 
situe un objet dont l'image est réellement ponctuelle sur le plan de la pellicule, puis la profondeur de champ.
\item Pour une focale $f'_{obj}=\unit{50}{\milli\meter}$, quelle est la distance minimale théorique entre l'objet pris en photo et l'objectif\,? Sachant qu'au 
maximum il est possible d'avancer l'objectif de $\unit{5}{\milli\meter}$ par rapport à la pellicule, quelle est la distance minimale à laquelle il est 
réellement possible de prendre une photo (on considérera que l'objet doit être parfaitement net)\,?
\item La tache centrale de diffraction donnée par une ouverture circulaire de dimaètre $D$ a pour rayon angulaire $\alpha=\frac{1,22\lambda}{D}$. Quelle
condition doit respecter le nombre d'ouverture $N=\frac{f'_{obj}}{D}$ d'un objectif de $\unit{50}{\milli\meter}$ de focale pour que la netteté ne soit pas 
limité par la diffraction\,? Faire l'application numérique pour $\varepsilon=\unit{30}{\micro\meter}$ et $\lambda=\unit{0,6}{\micro\meter}$.
\end{enumerate}
\pagebreak
\section{Développement des photographies}

Une pellicule photographique est recouverte d'une gélatine contenant des cristaux ioniques $AgBr$. Lorsque la lumière vient frapper la pellicule, elle arrache
des électrons aux ions bromures et ceux-ci sont captés par les ions argents. Il se forme donc des atomes d'argent solide (qui noircissent à la lumière).

Le but du développement est d'accélérer cette réaction pour les cristaux qui ont déjà partiellement réagit (en effet, après la très brève exposition, 
l'image n'est pas encore visible sur la pellicule), pour obtenir le négatif.

Pour procéder au développement photographique d'un film noir et blanc, on utilise un révélateur qui contient, entre autres, les composés suivants : de 
l'hydroquinone ($C_6H_6O_2$), du carbonate de potassium ($K^+,CO_3^{2-}$), du bromure de potassium ($K^+,Br^-$), du sulfite de sodium ($Na_2SO_3$) et de l'eau.

\begin{enumerate}
\item On considère que le $pH$ du révélateur est imposé par l'ion carbonate $CO_3^{2-}$. On donne pour le couple $HCO_3^-/CO_3^{2-}$ $K_a=6,4.10^{-11}$.
\begin{enumerate}
\item Ecrire la réaction de l'ion carbonate avec l'eau.
\item Calculer le $pH$ du révélateur en ne faisant intervenir que cette réaction (on néglique l'hydrolyse de l'ion hydrogénocarbonate).
\item Pouvait-on approximativement prévoir cette valeur de pH sachant que l'ion $CO_3^{2-}$ est majoritaire dans la solution de révélateur\,? 
Justifier votre réponse en traçant le diagramme de prédominance des ions pour le couple $HCO_3^-/CO_3^{2-}$
\end{enumerate}
\item Le réducteur utilisé dans le révélateur est l'hydroquinone $(C_6H_6O_2)$
\begin{enumerate}
\item Ecrire les demi-équations d'oxydo-réduction des couples $Ag^+/Ag$ et $C_6H_4O_2/C_6H_6O_2$.
\item En déduire la réaction qui se produit entre le révélateur et la pellicule
\end{enumerate}
\item On note $K_s$ La constante d'équilibre de la réaction $AgBr_{(s)} \rightleftharpoons Ag^+_{(aq)}+Br^-_{(aq)}$.

On donne $pK_s=12,5$.

Le révélateur contient également du bromure de potassium $(K^+_{(aq)}, Br^-_{(aq)})$.

On appelle solubilité d'un solide ionique - appelé soluté - la concentration maximale de ce composé que l'on peut dissoudre dans un solvant à une température
donnée. La solution obtenue est alors saturée. Ici on aura $s=[Ag^+]$.
\begin{enumerate}
\item Calculer la solubilité $s$ de AgBr dans l'eau pure.
\item La solubilité $s'$ de AgBr dans la solution de révélateur est plus faible que $s$
\begin{enumerate}
\item Expliquez pourquoi.
\item Calculez cette valeur $s'$.
\item En déduire pourquoi le bromure de potassium présent dans le révélateur s'appelle un anti-voile.
\end{enumerate}
\end{enumerate}
\end{enumerate}

\end{document}
