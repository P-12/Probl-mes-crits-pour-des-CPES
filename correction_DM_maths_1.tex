\documentclass[10pt,a4paper]{article}
\usepackage[utf8]{inputenc}  
\usepackage[T1]{fontenc}       
\usepackage[francais]{babel}
\usepackage{fullpage}
%\usepackage{euler}
\usepackage{amsmath}
%\usepackage{framed}
\usepackage{amsfonts}
\usepackage{amssymb}
\usepackage{pifont}
\usepackage{mathrsfs}
\usepackage{graphicx}
\usepackage[a4paper]{geometry}
\geometry{hscale=0.83,vscale=0.87,centering}

\pagestyle{empty}

\newcommand{\Titre}[3]{\begin{center} {\LARGE\textbf{\textsc{#1}}}\\ #2 \hfill \emph{#3} \\  \hrule\vspace{\baselineskip}\end{center}}


\begin{document}


%\title{Correction du DM de Maths}
%\date{18 novembre 2013}
%\author{Asp. J.Buet}
%\maketitle{}

\Titre{Correction du DM de Maths}{Asp J.Buet}{3 décembre 2013}
\thispagestyle{plain}
\pagestyle{plain}

\section{Etude de suites et de fonctions}
\begin{enumerate}
\item $u_1(1/2)=3/2, u_2(1/2)=7/4, u_1(-1)=0, u_2(-1)=1$.
\item 
\begin{enumerate}
\item La suite $(v_k(x))_{k\in\mathbb{N}}$ est une suite géométrique de raison $x$.
\item Une suite géométrique est convergente si sa raison est comprise dans $]-1,1]$. Donc la suite $(v_k(x))_{k\in\mathbb{N}}$ est convergente si $x\in]-1,1]$.
Elle a pour limite $0$ si $x\in ]-1,1[$ et pour limite 1 si $x=1$.
\item Si la suite $(v_k(x))$ ne converge pas, la suite $(u_n(x))$ ne peut pas converger. De plus, si $x=1$, la suite $(v_k(x))$ est constante égale à $1$ donc
la suite $(u_n(x))$ tend vers $+\infty$. Finalement, la suite $(u_n(x))$ ne peut converger que si $x\in]-1,1[$. Dans le cas général, quand une suite $(w_n)$
 s'écrit comme la somme des termes d'une autre suite $(c_n)$, la somme ne peut avoir une limite finie que si la suite $(c_n)$ tend vers $0$. Cette condition 
n'est pas suffisante (par exemple, $\sum\limits_{k=0}^n\frac{1}{k}$ est divergente).
\end{enumerate}
\item La formule sur la somme des termes d'une suite géométrique nous donne, si $x\neq1$ : $u_n=\frac{v_0-v_{n+1}}{1-x}$. On a donc $u_n=\frac{1-x^{n+1}}{1-x}$.
Si $x=1$ alors $u_n=n+1$.
\item On a déjà vu que la suite $(u_n(x))$ ne converge pas si $x=1$. Quand $x\neq1$, il suffit de regarder la limite de $x^{n+1}$. Si celle-ci existe et 
est finie alors $(u_n(x))$ converge. Donc $(u_n(x))$ converge si $x\in]-1,1[$.
\item Dans ce cas, on a $\lim\limits_{n\to+\infty}u_n(x)=\frac{1}{1-x}$.
\item Pour que $f$ soit définie il faut avoir $1-x\neq 0$. Donc le domaine de définition de $f$ est $\mathbb{R}\backslash\{1\}$.
\item Une autre expression de $g$ est donnée par la suite étudiée précédemment $g(x)=\sum\limits_{n=0}^{+\infty}x^n$ (la somme infinie n'est qu'une autre 
notation pour $\lim\limits_{n\to+\infty}\sum\limits_{k=0}^nx^k$). Cette nouvelle expression ne convient pas
 pour $f$ car les domaines de définitions ne sont pas les mêmes. La nouvelle expression de $g$ n'est valable que sur $]-1,1[$ alors que $f$ est définie sur
 tout $\mathbb{R}\backslash\{1\}$.
\end{enumerate}

\section{Autour de la convergence}
\begin{enumerate}
\item La suite $(t_n(x))$ est une somme de termes positifs. $t_{n+1}(x)-t_n(x)=|a_{n+1}x^{n+1}| \geqslant 0$.
\item Montrons que $(t_n(1/2))$ converge. La suite $(a_n)$ est convergente donc bornée : $\exists M \in \mathbb{R}^+, | a_n | \leqslant M$. Donc on 
peut majorer $t_n(x)$ par $M\cdotp\sum\limits_{k=0}^n| x^k |$. Cette suite majorante est à nouveau croissante, elle est donc plus petite que sa limite.
$$0\leqslant t_n(1/2)\leqslant M\cdotp\sum\limits_{k=0}^n| \frac{1}{2} | ^k \leqslant 2M$$
Donc $(t_n(1/2))$ est croissante majorée donc convergente. $x_0=1/2$ convient.
\item On remarque d'abord que $\alpha_n^+\geqslant 0$ et $\alpha_n^-\geqslant 0$. On remarque également que $\alpha_n^+$ et $\alpha_n^-$ ne sont pas
strictement positifs en même temps. Si $\alpha_n\geqslant 0$ alors $\alpha_n=\alpha_n^+$, sinon $\alpha_n=-\alpha_n^-$. On a donc bien
 $\alpha_n=\alpha_n^+-\alpha_n^-$. Puisque $| \alpha_n | = -\alpha_n=\alpha_n^-$ quand $\alpha_n \leqslant 0$,
on a bien $| \alpha_n | = \alpha_n^+ +\alpha_n^-$.
\item On a $\sum\limits_{k=0}^n| \alpha_k|=\sum\limits_{k=0}^n\alpha_k^+ +\sum\limits_{k=0}^n\alpha_k^-$. Les deux suites $(\sum\limits_{k=0}^n\alpha_k^+)$ et
 $(\sum\limits_{k=0}^n\alpha_k^-)$ sont croissantes positives. Donc on a $0\leqslant \sum\limits_{k=0}^n\alpha_k^+ \leqslant \sum\limits_{k=0}^n| \alpha_k |
 \leqslant \lim\limits_{n\to+\infty}\sum\limits_{k=0}^n | \alpha_k|$. Cette limite étant un réel positif, la suite $(\sum\limits_{k=0}^n\alpha_k^+)$
 est croissante majorée donc elle converge. Selon un raisonnement analogue, la suite $(\sum\limits_{k=0}^n\alpha_k^-)$ est croissante majorée donc convergente.
\item $\sum\limits_{k=0}^n\alpha_k=\sum\limits_{k=0}^n\alpha_k^+-\sum\limits_{k=0}^n\alpha_k^-$. Or si $(\sum\limits_{k=0}^n| \alpha_k|)$ converge, les deux suites 
$(\sum\limits_{k=0}^n\alpha_k^+)$ et $(\sum\limits_{k=0}^n\alpha_k^-)$ convergent. Donc la suite $(\sum\limits_{k=0}^n\alpha_k)$ converge et admet pour limite
 la différence des deux limites des deux suites $(\sum\limits_{k=0}^n\alpha_k^+)$ et $(\sum\limits_{k=0}^n\alpha_k^-)$.
\item $Re(b_n)=\beta_n$, alors le terme général de la suite $(\sum\limits_{k=0}^n|\beta_k|)$ est majoré par $\sum\limits_{k=0}^n|b_k|$. Or la suite 
$(\sum\limits_{k=0}^n|\beta_k|)$ est croissante (car somme de termes positifs) et majorée par la limite (réelle) de la suite  $(\sum\limits_{k=0}^n|b_k|)$
que l'on a supposée convergente. Donc la suite  $(\sum\limits_{k=0}^n|\beta_k|)$ converge. Selon un raisonnement analogue, on montre que
$(\sum\limits_{k=0}^n|\gamma_k|)$ converge. On se ramène ensuite à la question précédente (ce sont des suites réelles), on a donc les suites 
 $(\sum\limits_{k=0}^n\beta_k)$ et  $(\sum\limits_{k=0}^n\gamma_k)$ qui convergent, ce qui équivaut à dire que la suite $(\sum\limits_{k=0}^nb_k)$ converge 
d'après la propriété rappelée dans l'énoncé.
\item D'après les question précédentes, $(t_n(x_0))$ converge et en remplaçant $b_k$ par $a_kx^k$ dans la question précédente, on en déduit donc que 
$(f_n(x_0))$ converge.
\end{enumerate}

\section{Disque de convergence}
\begin{enumerate}
\item Soit $x\in\mathbb{R}, | x|<| x_0|$. La suite $(\sum\limits_{k=0}^n| a_kx_0^k|)$ converge donc $(a_kx_0^k)$ converge 
(et a pour limite $0$) donc elle est majorée, notons $M$ un tel majorant.
$\sum\limits_{k=0}^n| a_kx_0^k|.|\frac{x}{x_0}|^k\leqslant M.\sum\limits_{k=0}^n| \frac{x}{x_0}|^k$.
On reconnaît la somme d'une suite géométrique de raison $| \frac{x}{x_0}|$. Or $| x| < |x_0|$, donc la somme de la suite géométrique converge.
La suite $(\sum\limits_{k=0}^n| a_kx^k|)$ est donc majorée par $M.\frac{1}{1-|\frac{x}{x_0}|}$ et elle est croissante (somme de termes positifs), donc elle 
converge. Donc finalement $(t_n(x))$ converge. D'après la partie précédente, on a donc $(f_n(x))$ converge.
\item Le raisonnement est exactement analogue pour $z \in \mathbb{C}$, on n'a utilisé à aucun moment le fait que $x$ était réel, on raisonne de la même 
manière avec le module au lieu de la valeur absolue et la partie précédente nous permet de conclure dans le cas d'une suite complexe.
\item Puisque la fonction $f$ de la première partie n'est pas définie en $1$, on ne peut au maximum ne trouver une somme de termes de la forme $a_kx^k$
qui correspond à $f$ que sur 
l'intervalle $]-1,1[$. En effet, si on trouvait une telle somme qui définissait $f$ en $-2$, d'après ce que l'on vient de montrer, elle serait définie sur
tout l'intervalle $[-2,2]$, donc également en $1$, ce qui est impossible. En réalité, c'est même les valeurs interdites de la fonction définies sur les 
complexes qui rentrent en jeu. En effet, si la fonction définie à l'aide de la somme existe en $1$, elle existe sur tout le disque de centre $0$ et de rayon 
$1$ (en considérant le plan complexe). Ainsi, la fonction $x \mapsto \frac{1}{1+x^2}$ ne s'annule pas sur $\mathbb{R}$ mais s'annule en $i$. Donc si l'on 
cherche à l'exprimer à l'aide d'une somme de termes de la forme $a_kx^k$, cette expression ne sera au mieux valable que sur l'intervalle réel $]-1,1[$ (et 
sur le disque ouvert complexe de centre $0$ et de rayon $1$).

Lorsque l'on cherche ainsi à écrire une fonction comme limite d'une somme de termes de la forme $a_kx^k$, on dit que l'on cherche à développer cette fonction
en série entière. C'est un problème complexe car comme on vient de le voir, le domaine de définition de la somme et de la fonction ne sont pas forcément
les mêmes. Il n'est pas non plus toujours possible d'écrire une fonction d'une telle manière.
\end{enumerate}

\end{document}
